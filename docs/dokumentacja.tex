\documentclass{article}
\usepackage[utf8]{inputenc}
\usepackage{amsmath}

\title{DOKUMENTACJA}
\author{}
\date{26 stycznia 2025r.}

\begin{document}

\maketitle

\section{Co to}
Projekt OCRaider to aplikacja webowa, która umożliwia konwersję dokumentów w formacie PNG na czytelny tekst. Aplikacja wykorzystuje algorytmy rozpoznawania tekstu (OCR) do przetwarzania obrazów i wydobywania z nich informacji tekstowych.

\section{Jak działa}
Aplikacja składa się z dwóch głównych części: frontend i backend. Frontend jest odpowiedzialny za interfejs użytkownika, w którym użytkownicy mogą przesyłać pliki do przetworzenia. Backend obsługuje logikę przetwarzania plików oraz komunikację z algorytmem OCR.

\begin{enumerate}
    \item Użytkownik przesyła plik graficzny za pomocą interfejsu frontendowego.
    \item Plik jest wysyłany do backendu, gdzie jest przechowywany tymczasowo.
    \item Backend przetwarza plik za pomocą algorytmu OCR, który analizuje obraz i wydobywa tekst.
    \item Przetworzony tekst jest zwracany do frontendu, gdzie użytkownik może go zobaczyć i skopiować.
\end{enumerate}

\section{Jak uruchomić aplikację}
Aby uruchomić aplikację, wykonaj poniższe kroki:

\begin{enumerate}
    \item **Klonowanie repozytorium**: Użyj poniższego polecenia, aby sklonować repozytorium na swój lokalny komputer:
    \begin{verbatim}
    git clone https://github.com/kazikd11/OCR/
    \end{verbatim}
    
    \item **Instalacja zależności**: Przejdź do katalogu frontend i backend, a następnie zainstaluj wszystkie zależności:
    \begin{verbatim}
    Do odpalenia wymagany jest zainstalowany node.js oraz python.

    cd OCRaider/client
    npm install
    cd ../server
    pip install -r requirements.txt
    \end{verbatim}
    
    \item **Uruchamianie serwerów**: Uruchom serwery frontendowy i backendowy w osobnych terminalach:
    \begin{itemize}
        \item Dla frontendu:
        \begin{verbatim}
        cd OCRaider/client
        npm run dev
        \end{verbatim}
        
        \item Dla backendu:
        \begin{verbatim}
        cd OCRaider/server
        python manage.py runserver
        \end{verbatim}
    \end{itemize}
\end{enumerate}

\section{Kto co robił}
\begin{itemize}
    \item \textbf{Frontend}: Michał Saturczak - odpowiedzialny za rozwój interfejsu użytkownika oraz integrację z backendem.
    \item \textbf{Integracja frontu z backendem}: Michał Kaźmierczak - zajmował się połączeniem frontendowej części aplikacji z backendem.
    \item \textbf{Algorytm OCR}: Mateusz Adamczyk i Patryk Hołubowicz - stworzyli algorytm rozpoznawania tekstu, który jest kluczowym elementem aplikacji.
\end{itemize}

\section{Co nie działa}
W aplikacji występuje problem z rozpoznawaniem małych liter. Algorytm OCR nie jest w stanie poprawnie przetwarzać tych znaków, co wpływa na jakość rozpoznawanego tekstu.

\end{document}